\chapter{結論}
\begin{large}
\begin{quote}
本章では,本システムの結論を述べる.
\end{quote}
\end{large}
\clearpage

\section{今後の課題と展望}
%本研究では,センサデータの時間的特殊性に着目することにより,保存ピア探索における計算コスト,データ保存に要する時間,データ取得に要する時間について,既存の手法と比較し,それぞれ良い知見が得られた.しかし,データの取得に要する時間に関しては,依然として,課題が残る.既存手法との比較では,時間短縮に成功したが,半径10の領域から100個のデータの取得を行うに際し,時間数万msの時間を要している.想定するシナリオに挙げたように,ユーザがスマートフォンから情報を取得し,リアルタイムにアプリケーションに反映させることを斟酌すると,この所要時間では到底満足できない.よって,今後は,このデータの取得に着目した研究が必要となる.これに関する指針として,センサデータ管理のマルチレイヤ化がある.現行のT-Ringシステムは,全てのデータを生データのまま保存している.しかし,現実に取得されるデータは,10:00,11:00など切りの良い時間から取得される可能性が高いと考え得る.また,値の平均や最大,最小などが利用されるケースも存在する.このように,個々のセンサデータについての特徴は存在しないながらも,ある一定のデータが集約されることにより,特定の意味を有する1つのデータになることは十分に考えられる.これらの,データの集約による特徴ごとに,マルチレイヤで管理することにより,ユーザのクエリに対する対象データ数の削減に寄与すると考えられる.

%また,本研究のメインフォーカスとして挙げてはいないが,近年,Cyber-Physical Systemsと呼ばれる,実空間情報を情報空間に取り込むことにより,新たな価値を提供することを目指す研究分野が注目されている.Cyber-Physical Systemsでは,人間や物の行動や動きを逐次記録する手法が用いられることがある.この記録データは,センサデータと同様に時間に伴って増大する.T-Ringはこのようなデータを管理する手法としても用いることが可能である.

\section{本論文のまとめ}
%本研究は,従来の多次元データ管理手法がセンサデータにおける時間のような,パラメータに特徴のあるデータを対象とした手法が存在しないことに着目した.次いで,センサデータの時間的特殊性に着目したT-Ringシステムの提案を行った.このシステムの評価を行った結果,既存のシステムと比べ,データ保存,取得において,高速化を実現した.しかし,データの取得に関しては,リアルタイムアプリケーションに利用するに耐えうる所要時間ではないので,最終章において,今後の課題として取り組むべきであることに言及した.
