\chapter{結論}
\begin{large}
\begin{quote}
本章では,本システムの結論を述べる.
\end{quote}
\end{large}
\clearpage

\section{今後の課題と展望}
%より厳密なリアルタイム処理を行うために,
%各ノード間での時刻同期を行う
無線センサネットワークは複数の無線センサノードによって構成される分散システムであるため,
本研究で想定しているような
環境モニタリングやターゲットトラッキングなどの
時間的整合性が要求されるアプリケーションの構築には,
何らかの手段によって時刻同期を行うことが求められている.
代表的なオペレーティングシステムである
TinyOS\cite{Hill:2000:SAD:356989.356998}\cite{Levis04tinyos:an}における
Flooding Time Synchronization Protocol(FTSP)\cite{Maroti:2004:FTS:1031495.1031501}
と呼ばれる高精度な時刻同期プロトコルでは,
平均約1$\mu$sの時刻同期の実現に成功している.
しかしながら,
Contiki\cite{Dunkels:2004:CLF:1032658.1034117}が
独自に採用している時刻同期プロトコルの
精度はFTSPと比較してかなり劣っているのが現状である.
また,通常時は無線を休止し,
無線の起動を一定に保つことに加えて,
受信すべきデータがネットワークに存在しなければ
すぐに無線を休止状態に戻す
MACプロトコルが提案されているが,
高精度な時刻同期をすることにより
無線の起動時間を短くすることができるため,
消費電力を抑えることも可能となる.
したがって今後の展望として,
Contikiにおける高精度な時刻同期プロトコルを実装することで,
より厳密なリアルタイム処理ができるようにすることを目指す.

また,Contikiは
MicaZ\cite{Hill:2002:MWP:623308.624560}以外にも,
Iris Mote\cite{irismote}や
Tmote Sky\cite{tmote_sky}などの
多様なプラットフォームへの対応が確認されているが,
現状として本システムは今回実装として採用したMicaZ上でしか
動作を確認することができていない.
CPUのアーキテクチャが異なるプラットフォームに
移植した際に,
正常に動作できるようにすることも今後の課題として挙げる.
%今後CPUのアーキテクチャが異なるプラットフォームでも
%正常に動作できるようにする.



\section{本論文のまとめ}
%本研究は,従来の多次元データ管理手法がセンサデータにおける時間のような,パラメータに特徴のあるデータを対象とした手法が存在しないことに着目した.次いで,センサデータの時間的特殊性に着目したT-Ringシステムの提案を行った.このシステムの評価を行った結果,既存のシステムと比べ,データ保存,取得において,高速化を実現した.しかし,データの取得に関しては,リアルタイムアプリケーションに利用するに耐えうる所要時間ではないので,最終章において,今後の課題として取り組むべきであることに言及した.

本論文では,
時間的制約の伴った環境モニタリングや
ターゲットトラッキング時における,
省エネルギーなリアルタイム処理の必要性について述べ,
既存のオペレーティングシステムを採用した際の問題点について言及した.
次いで,
無線センサネットワークにおけるオペレーティングシステムに対する
動的なスケジューリングスイッチング機構
であるD-Switchを提案し,
通常時はイベントモデルとしてイベントに応じた処理を行うとともに,
リアルタイムイベント発生時には
現在のタスクを中断し,
リアルタイムタスクの実行が完了し次第,
中断したタスクの処理を再開するシステムを設計し,評価を行った.
実験の結果,本システムを拡張する前のContikiにおける電力消費を
大幅に増加させることなく,
82ms以内のタスクの実行を可能にした.



