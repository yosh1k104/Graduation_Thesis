\chapter*{謝辞}
本論文の執筆にあたり,親身になって丁寧にご指導して頂きました,慶應義塾大学環境情報学部教授徳田英幸博士に深く感謝致します.
また,貴重なご助言を頂きました慶應義塾大学環境情報学部准教授高汐一紀博士,慶應義塾大学環境情報学部准教授中澤仁博士,慶應義塾大学環境情報学部米澤拓郎特任助教,
慶應義塾大学環境情報学部陳寅特任助教,
慶應義塾大学政策・メディア研究科研究員伊藤友隆氏,
慶應義塾大学政策・メディア研究科博士課程小川正幹氏
に深く感謝致します.

慶應義塾大学徳田研究室の諸先輩方には折に触れ貴重なご助言を頂き,また多くの議論の時間を割いて頂きました.
特に,政策・メディア研究科修士課程伊藤瑛氏,安形憲一氏には,本研究に対し,多くの時間を割いて頂きました.
ここに多大なる感謝と尊敬の意を表します.

また,
MEMSYS,
Link研究グループにおいて,
研究活動だけではなく,
公私に関わらず,
親しく接していただいただいた
江頭和輝氏,
早川侑太朗氏,
宮川眞海子氏,
佐田真穂氏,
井上美菜子氏,
寺山淳基氏,
皆川昇子氏,
荻野メリッサ氏,
池田貴匡氏,
神崎亜実氏,
水谷正芳氏
を始めとした諸先輩,
後輩方,
毎週のように食事をともにした榊原寛氏,
日吉ハンドボール同好会に所属する皆様,
立川高校ハンドボール部OBの皆様
に深く感謝致します.

最後に,
数少ない同学年として,
研究活動に切磋琢磨した古川侑紀氏,
卒論提出が迫り,氷点下付近まで気温が下がっているにも関わらず,ユニークなギャグで
より一層周りの気温を低下させ,さらなる精神修行を課してくれた
宇佐美真之介氏,
嫌いと言われつつも,残留生活においておそらく一番長く時間を共有し,
数々の罵詈雑言を浴びせるという精神修行を課してくれた
高木慎介氏,
毎週のように楽しみにしていた週刊少年ジャンプのネタバレをいつも楽しそうに語り,
生きていく目標とは何かを考えさせてくれるような精神修行を課してくれた
豊田智也氏に深く感謝し,
謝辞と致します.




%%%%%%%%%%%%%%%%%%%%%%%%%%%%%%%%%%%%%%%%%%%%%%%%%%%%%%%%%%%%%%%%%%%%%%%%%%%%%%%%
%本研究の機会を与えてくださり,絶えず丁寧なご指導を賜りました,慶應義塾大学環境情報学部教授徳田英幸博士に深く感謝致します.また,貴重なご助言を頂きました應義塾大学環境情報学部准教授高汐一紀博士に深く感謝致します.
%また,慶應義塾大学徳田研究室の諸先輩方には折に触れ貴重なご助言を頂き,また多くの議論の時間を割いて頂きました.特に政策メディア研究科修士課程小川正幹氏,今枝卓也氏には,本論文の執筆にあたってご指導を頂きました.また政策メディア研究科講師中澤仁博士には本研究を進めるにあたって多くの励ましとご指導を頂きました.ここに深い感謝の意を表します.
%最後に,研究生活を経済的にだけでなく,精神的にも支えてくれた家族,研究の日々を家族同然に同じ時間を共に過ごした慶應義塾大学総合政策学部4年天野雅哉氏,慶應義塾大学環境情報学部4年米川賢治氏,モヒカンの頃が懐かしい井村和博氏,日本,研究の日々を共に過ごしたACE 研究グループの勝治宏基氏,その他多くの友人に深く感謝し,謝辞と致します.
\begin{flushright}
\today\\
小町 芳樹
\end{flushright}
