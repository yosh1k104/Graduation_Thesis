\begin{center}
\textbf{\Large 卒業論文要旨 2014年度(平成26年度)}

\vspace{6.18mm}

\textbf{\Large D-Switch: 無線センサネットワークにおけるオペレーティングシステムに対する動的なスイッチング機構}
\end{center}

\vspace{10mm}
近年技術の発展によりセンサノードの低価格化,高性能化が進み,
それに伴い,ネットワークに繋がる物理センサが自動的に多様なデータをやり取りし,
それらを様々な形で活用する無線センサネットワークが普及しつつある.
主にその結果は,環境モニタリングやターゲットトラッキングにおいて
顕著に表れており,センサネットワークが我々を支える生活基盤となり得る将来は遠くはない.

無線センサネットワークにおけるオペレーティングシステムは
イベントモデルとスレッドモデルに分類できるが,
%環境モニタリングやターゲットトラッキングなどの
リアルタイム性のある処理が
長期間にわたって行われるような環境では,
それらのモデルのどちらかを選択することは望ましくない.
Dunkelsらの開発したContiki\cite{Dunkels:2004:CLF:1032658.1034117}
では,
Protothreads\cite{Dunkels:2006:PSE:1182807.1182811}
というシステムを採用することで,
%イベントモデルでありながらもスレッドモデルのようなプログラムの記述を実現している.
イベントモデルでありながらもスレッドモデルと同様の挙動を行うことができる.
しかしながら,Protothreadsを採用した場合でも,
タスク実行中にさらに優先度の高いタスクの実行を要請された際に,
スレッドモデルのように割り込みをし,タスクを切り替えることはできないのが現状である.

本研究では,Protothreadsを利用し,
上記ような時間的制約を伴ったイベントが発生する環境における,
省電力なオペレーティングシステムについて提案する.
本システムは通常イベントモデルとしてタスクの実行を行うことで省電力を実現し,
リアルタイム性の必要なタスク発生時には割り込みを発生させ,
優先的にそのタスクの処理を行うことを可能にしている.
最後に本システムの性能を評価し,有用性を証明する.


\vspace{10mm}
キーワード :\\
\hspace{3.5em}無線センサネットワーク,環境モニタリング,ターゲットトラッキング,オペレーティングシステム,リアルタイム処理

\begin{flushright}
\textbf{慶応義塾大学環境情報学部}\\
\textbf{小町 芳樹}
\end{flushright}

\newpage

\begin{center}
\textbf{\Large Abstract of Bachelor's Thesis}
\textbf{\Large Academic Year 2014}
\vspace{6.18mm}

\textbf{\Large D-Switch: A Dynamic Switching System for Operating Systems in Wireless Sensor Networks}
\end{center}

\vspace{10mm}

In recent years, 
developed technology made 
sensor nodes cheaper and high performance.
As a result of this advanced technology, 
Wireless Sensor Networks (WSN) 
in which physical sensors connected with networks
automatically transmit and receive various data 
are constantly growing and used in various ways.
In particular, the results are most remarkable
in environmental monitoring and target tracking.
In the future, WSN will become the foundation of our life supporting. 

Operating Systems (OS) in WSN
are divided into events model and threads model.
However, it is undesireble to 
select OS from these models  
in environments where 
real time tasks have been processed 
for a long period.
%for a long period,
%such as environmental monitoring and target tracking.
We can program applications easily
with OS
in events model which
has the characteristics of threads model
by using Protothreads\cite{Dunkels:2006:PSE:1182807.1182811} 
in Contiki\cite{Dunkels:2004:CLF:1032658.1034117}
proposed by Dunkels et al.
However, 
if a task which has more priority was posted 
while another task has been processed,
the processed task cannot be interrupted,
%and we cannot switch the processed task to the arrived task 
and the processed task cannot be switched to the arrived task 
with Protothreads
in the existing circumstances.

In this research, we propose an energy-efficient operating system 
using Protothreads
in environments where events with time constraint occurred.
In this system,
tasks are usually executed as general events model OS, 
which saves energy.
If an event which requires real time processing occurred,
current task will be interrupted and 
real time task will be dealt with as priority.
Finally, we evaluate this system and prove feasibility.


\vspace{10mm}
Keywords :\\
\hspace{3.5em}Wireless Sensor Networks, Environmental Monitoring, Target Tracking, Operating Systems, Real Time Processing 
\begin{flushright}
\textbf{Yoshiki Komachi}\\
\vspace{5mm}
\textbf{Faculty of Environment and Information Studies}\\
\textbf{Keio University}
\end{flushright}
