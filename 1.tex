\chapter{序論}
\begin{large}
\begin{quote}
%本章では,まずセンサデータの増加などの本研究の背景を述べ,次いで,センサデータの公衆一般化において発生する問題の解決という本研究の目的について言及し,最後に本論文の構成を示す.
本章では,まず
無線センサネットワークオペレーティングにシステム
関する本研究の背景を述べ,
次いで,
%センサデータの公衆一般化において発生する問題の解決という
%オペレーティングシステムのモデル選択において
%発生する問題の解決という
本研究の目的について言及し,
最後に本論文の構成を示す.
\end{quote}
\end{large}
\clearpage

\section{本研究の背景}
近年センサの低価格化,高性能化により,センサネットワークが普及し,
それに伴いセンサ用のオペレーティングシステムの研究も発展を遂げている.
センサネットワークのオペレーティングシステムには主に2種類あり,
省資源で低消費電力を実現したイベントモデルと,
リアルタイム処理を可能としたスレッドモデルが存在している.
センサネットワーク用のオペレーティングシステムではイベントモデルが主流となっている.

イベントモデルのTinyOS\cite{Hill:2000:SAD:356989.356998}\cite{Levis04tinyos:an}
は,省資源で低消費電力の実現に成功しているが,
リアルタイム処理を行うことができておらず,
それに加えて
ユーザが一連の処理を細かい処理に分割しなければならないため,
プログラムが書き辛い.
%また,言語もnesCという特殊な言語を使用しているため,
%プログラムが非常に書きにくいという欠点も存在する.
それに対して,Nano-RK\cite{Eswaran:2005:NER:1106608.1106672}
はリアルタイム処理をサポートしているが,
イベントモデルのオペレーティングシステムと比較すると資源の消費が大きいため,
電力の消費が激しい.

Contiki\cite{Dunkels:2004:CLF:1032658.1034117}
はイベントモデルを採用したオペレーティングシステムで
%,使用言語はC言語である.
%ContikiはProtothreadsと呼ばれるメカニズムを使用しており,Contikiは
Protothreads\cite{Dunkels:2006:PSE:1182807.1182811} 
と呼ばれるメカニズムを使用しており,
%イベント駆動型でありながらスレッドの切り替えを行うことができる.
イベントモデルでありながらスレッドモデルと同様の挙動を行うことができる.
%そのため,イベントモデルの短所であるプログラムの書きにくさを解消することに成功している.
そのため,イベントモデルの短所であるプログラムの書きにくさを解消しつつ,
実行中のタスクの切り替えを可能にしている.
しかし,タスクを処理する優先順位を明確に決める手法が定められていないため,
タスクの切り替えはできるが,未だにリアルタイム処理は不可能なのが現状である.


\section{本研究の目的}
Contikiはイベントモデルでありながら,
マルチスレッドでタスクを変更することができるため,
リアルタイム処理と親和性があると推測できる.
本研究では,Contikiにおける動的なスケジューリングスイッチング機構を提案し,
省資源性で低消費電力を保ちながら,
リアルタイム処理の可能なオペレーティングシステムの実現を目的とする.
%最後に,本システムの評価を行うとともに,有用性を証明する.


%多次元データに対して,センサデータの時間的特殊性を考慮した,T-Ringという分散管理手法を提案し,管理について,利用するアプリケーションの特徴を幾つかの軸で分類し,既存の手法を用いたシステムがどのようなアプリケーションに適するのかを明らかにする.次に,センサデータの時間的特殊性を考慮したシステムが必要なアプリケーションを提示する.最後に,センサデータの時間的特殊性を考慮したシステムの提案し,パフォーマンスの低下を防ぐ.

\section{本論文の構成}
%本論文では,2章において,センサネットワークの一般公衆化とアプリケーションの分類から,そのデータ管理において必要な機能要件を示し,センサデータの時間的特殊性について述べる.3章では,センサデータではない公衆コンテンツの管理とDHTを用いた広域公衆センサデータについての関連研究を紹介する.4章,5章では,センサデータの時間的特殊性に着目したデータ管理システムを示す.6章で,提案手法に対する評価を行い,7章で本研究のまとめを行う.

本論文では,
2章において,無線センサネットワークにおけるアプリケーションを分類し,
それぞれにおける特徴について議論する.
3章では,無線センサネットワークにおけるオペレーティングシステムについて紹介し,
各システムについて比較を行う.
4章で,無線センサネットワークにおけるモデル選択に関する問題点について言及し,
5章では,その問題を解決する手法について,設計と実装の観点から考察する.
6章にて,提案手法に対する評価を行い,7章で本研究のまとめを行う.

